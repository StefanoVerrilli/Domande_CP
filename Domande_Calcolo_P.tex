\documentclass[20pt]{article}
\usepackage[a4paper, total={6in, 9in}]{geometry}
\usepackage{xcolor}
 \renewcommand{\familydefault}{\rmdefault}
 \newcommand{\RomanNumeralCaps}[1]
 {\MakeUppercase{\romannumeral #1}}

\begin{document}

\section{Domande di preparazione Calcolo Parallelo 2021:}
\begin{enumerate}
\color{black} \item Nella formulazione: $k \cdot T(n) \cdot \mu$, cosa rappresentano T(n) e $\mu$ e perché presentano entrambi dei limiti di ottimizzazione?
\color{black} \item Descrivi brevemente i tre tipi di parallelismo
\color{black} \item Quale differenza c'è tra processore e core?
\color{black} \item Qual è la differenza tra Calcolo Parallelo in ambiente Distribuito \textit{(DM)} e in ambiente condiviso \textit{(SM)}?
\color{black} \item Come è possibile ricavare la complessità di tempo con \textit{p} processori per la somma di \textit{n} numeri utilizzando la seconda strategia?
\color{black} \item Quando è necessario utilizzare la formulazione della \textbf{legge di Ware-amhdal generalizzata} invece di quella normale? 
\color{black} \item Nella formulazione classica della legge di Ware-amhdal, cosa rappresenta l'$\alpha$ e l'$1 - \alpha$?
\color {black} \item Perché per l'algoritmo della somma, all'aumentare dei \textit{p} processori l'efficienza diminuisce drasticamente superati i 2?
\color{black} \item Nel nucleo computazionale della somma di \emph{n} numeri con \emph{p} processori, a quanto equivale la complessità di tempo dell' algoritmo (utilizzando la seconda strategia di collezione dei risultati) in ambiente MIMD-DM e MIMD-SM?
\color{black} \item Per quale motivo non è necessario sviluppare una terza strategia per la collezione, in ambiente MIMD-SM, della somma di \textit{n} numeri?
\color{black} \item Relativamente al numero di processori \textit{p}, quale caratteristica necessito per effettuare la collezione dei risultati tramite la \RomanNumeralCaps{2} strategia?
\color{black} \item Nell'algoritmo di somma di \textit{N} numeri, se ci trovassimo nel caso in cui il numero di elementi non è perfettamente divisibile per il numero di processori \textit{p}, come dovremmo procedere?
\color{black} \item In che modo la non esatta divisibilità del numero di elementi per il numero di processori inficia negativamente per il calcolo dello \emph{speed-up}?
\color{black} \item Quale rapporto esiste tra il tempo di comunicazione $t_{com}$ e il tempo di svolgimento di un operazione floating point $t_{calc}$?
\color{black} \item Cosa comporta la clausola \textit{firstprivate} se utilizzata all'interno del costrutto \textit{parallel} della libreria OpenMP?
\color{black} \item Cosa comporta la clausola \textit{lastprivate} della libreria OpenMP e con che costrutto viene utilizzato?
\color{black} \item Perché il nucleo computazionale del prodotto matrice-vettore con la prima strategia di suddivisione viene definito del tipo \emph{embarassing parallel}?
\color{black} \item Perché è necessario per lo speed-up scalato tener traccia anche della dimensione \textit{n} del problema?
\color{black} \item Nella clausola \underline{scheduling} della libreria OpenMP per il costrutto parallel, quando è preferibile utilizzare la keyword \textit{dyncamic} invece di \textit{static}?
\color{black} \item Quali sono i costrutti \underline{worksharing} della libreria OpenMP? Sapresti definirne i loro differenti utilizzi?
\color{black} \item A che valori corrisponde \emph{l'efficienza ideale} e \textit{lo speed up ideale} all'interno di un nucleo computazionale?
\color{black}\item Come è possibile ottenere il nuovo numero di elementi \textit{$n_1$} da sommare dato il nuovo numero di processori \textit{$p_1$} tale che il nostro problema sia scalabile?
\color{black} \item Per poter affermare che il problema sia \underline{scalabile} quale condizione deve essere posta sull'efficenza \textit{E}?
\color{black} \item Riusciresti a ricavare l'overhead per quanto riguarda la prima, seconda e terza strategia del prodotto matrice-vettore in ambiente MIMD-DM?
\color{black} \item Per quanto riguarda la seconda strategia del prodotto matrice-vettore, come viene assegnato il vettore \textit{b} ad ognuno dei core/processori?
\color{black} \item Definisci l'overhead totale e dimostra come questo sia nullo per la prima strategia del nucleo computazionale del prodotto matrice-vettore.
\color{black} \item Nella terza strategia del prodotto matrice-vettore, da quale misura dipende il numero di operazioni fatte nella collezione dei dati?
\color{black} \item Come è possibile definire il numero di thread \underline{all'interno} di un programma utilizzante la libreria OpenMP? Come è possibile invece definirlo dall'esterno?
\color{black} \item Come deve essere suddiviso il lavoro nella somma di \textit{n} elementi nel caso in cui non ci sia l'esatta divisibilità per il numero di processori \textit{p}?
\color{black} \item Con 8 processori ed 67 elementi da sommare, come è possibile calcolare lo speed-up utilizzando la formulazione generalizzata di Ware-amhdal con la seconda strategia di collezione?
\color{black} \item Nella terza strategia del prodotto matrice-vettore, nel caso in cui non ci sia l'esatta divisibilità sia per il numero di righe che di colonne, come si deve agire sulla suddivisione della matrice e del vettore? Come è possibile calcolarne lo speed-up tramite la formulazione di Ware-amhdal generalizzata?
\color{black} \item Il costrutto \underline{for} della libreria OpenMP necessita di parentesi all'apertura? Spiega il motivo della risposta.
\color{black} \item Nel nucleo computazionale del prodotto matrice per vettore, utilizzando la terza strategia della suddivisione dei dati e la seconda strategia per la collezione dei risultati, come è possibile ottenere la complessità di tempo con \emph{p} processori in ambiente MIMD-DM e MIMD-SM?
\color{black} \item Qual è la differenza tra la \emph{weak scalability} e la \emph{strong scalability} e da quali leggi vengono definite entrambe? 
\color{black} \item Cosa comporta al routine della libreria MPI \emph{Scatter} e come differisce dalla routine \emph{Broadcast} sempre di quest' ultima libreria?

\end{enumerate}
\end{document}
